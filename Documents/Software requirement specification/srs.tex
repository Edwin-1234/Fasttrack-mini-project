\documentclass{article}
\usepackage{graphicx} % Required for inserting images
\usepackage[left=2cm,top=2cm,right=2cm,bottom=2cm,bindingoffset=0.4cm]{geometry}
\begin{document}
\begin{center}
\textbf{\Huge Software Requirement Specification}\\
\vspace{70pt}
\textbf{\Large for}\\
\vspace{60pt}
\textbf{\LARGE FastTrack}\\
\vspace{40pt}
\textbf{\large Prepared by}\\
\vspace{30pt}
\textbf{\Large Akhil S Nair (KTE22CS009)}\\
\vspace{30pt}
\textbf{\Large Alwin Philip (KTE22CS012)}\\
\vspace{30pt}
\textbf{\Large Edwin Varkey (KTE22CS028)}\\
\vspace{30pt}
\textbf{\Large Jessin Sunny (KTE22CS036)}\\
\vspace{70pt}
\textbf{Department of Computer Science and Engineering}\\
\vspace{20pt}
\textbf{Rajiv Gandhi Institute of Technology,Kottayam}
\end{center}
\newpage
\tableofcontents
\newpage
\section{Introduction}
\subsection{Purpose}
\emph{This Software Requirements Specification document outlines the requirements for the FastTrack application, a platform designed to calculate and optimize delivery routes for delivery personnel efficiently. It specifies the functionalities, features, and requirements of the product. It covers aspects like location input, default location integration, route optimization algorithms, time estimation, and route visualization through an interactive map interface.  
The primary scope is to enhance the efficiency and reliability of delivery operations while minimizing travel time and fuel consumption. The application aims to provide an intuitive interface, seamless navigation, and support for real-time updates to improve delivery service quality and customer satisfaction.}
\subsection{Document Conventions}
\emph{
\begin{itemize}
    \item This document follows a standardized format, including numbered sections, bullet points, diagrams, and tables for clarity and readability.
    \item Acronyms, abbreviations, and technical terms are defined upon first usage and compiled in the glossary (Appendix A) for quick reference.
    \item All requirements are assigned unique identifiers and listed systematically to ensure easy tracking and cross-referencing throughout the document. 
\end{itemize}
} 
\subsection{Intended Audience and Reading Suggestions}
\emph{This document is intended for the following readers:
\begin{itemize}
    \item Developers: To understand the technical requirements and system functionality.
    \item Project Managers: To track progress and ensure the requirements are implemented.
    \item Testers: To create test cases based on requirements.
    \item End Users (Delivery Personnel and Administrators): To comprehend the application's functionality and benefits.
\end{itemize}
}
\emph{Reading Suggestions include
\begin{itemize}
    \item Overview Sections: Recommended for all readers to gain a general understanding of the application.
    \item Functional and Non-functional Requirements: Crucial for developers, testers, and project managers to ensure proper implementation and verification of the system.
    \item User Interface and Workflow Sections: Important for end users to familiarize themselves with the application's features and usage.
\end{itemize}}
\subsection{Project Scope}
\emph{
The FastTrack application is designed to streamline delivery operations by optimizing routes for delivery personnel. Its key objectives include:
\begin{itemize}
    \item Efficient Route Optimization: Calculating the most efficient delivery route for multiple destinations using advanced algorithms, reducing travel time and fuel consumption.
    \item Default Location Integration: Incorporating a fixed starting point for consistent and reliable route planning.
    \item Interactive Map Visualization: Providing clear and user-friendly route displays, along with real-time navigation support.
    \item Time and Distance Estimation: Offering accurate estimates for delivery time and travel distance to enhance scheduling and customer satisfaction.
    \item Dynamic Updates: Allowing for real-time adjustments to routes in response to changes, such as additional stops or delays.
\end{itemize}
The application aims to improve delivery efficiency, reduce operational costs, and ensure timely deliveries while enhancing the overall user experience for delivery personnel and service providers.
}
\subsection{References}
\emph{
\begin{itemize}
    \item Route Optimization Apps to Minimize Delivery Delays with Dynamic Routes, nandbox, 2024. Retrieved from nandbox.com
    \item Route4Me API and SDK Developer Documentation, Route4Me, 2024. Retrieved from support.route4me.com
    \item How to Use Python for Route Optimization in Logistics, Data Head Hunters Academy, 2024. Retrieved from dataheadhunters.com
    \item Delivery Route Optimization in Python, AskPython, 2024. Retrieved from askpython.com
\end{itemize}
}
\section{Overall Description}
\subsection{Product Perspective}
\emph{The FastTrack application is a new, self-contained product designed to address the challenges of delivery route optimization. Unlike existing manual or semi-automated systems, FastTrack leverages advanced algorithms and geolocation technology to create an intuitive, user-friendly platform for delivery personnel.
This application operates independently, without requiring integration with any legacy systems, but can seamlessly integrate with popular GPS and mapping tools. It is designed to serve small and medium-sized businesses seeking efficient delivery solutions, enhancing operational efficiency and reducing delivery time.
The concept stems from the need to simplify delivery operations, reduce costs, and improve the overall reliability of delivery services, particularly in dynamic environments with multiple destination points.}
\subsection{Product Features}
\emph{The primary feature of FastTrack include:
\begin{enumerate}
    \item Route Optimization: Users can efficiently plan routes with up to 300 stops, utilizing real-time traffic data and delivery-specific parameters for optimized navigation.
    \item Dynamic Route Adjustment: Users can adapt to changing conditions through real-time route adjustments, minimizing delays caused by traffic, road closures, or weather disruptions.
    \item Proof of Delivery and Documentation: Users can securely document proof of delivery through photos, notes, and signatures, ensuring seamless record-keeping and dispute resolution.
    \item Dynamic Data-Driven Delivery Insights: Users can access detailed delivery performance insights, including metrics like distance, time, and fuel savings, with exportable reports for analysis and improvement.
    \item Route Export and Data Integration: The system enables seamless export of route data in CSV or Excel formats, ensuring integration with logistics and operations management systems.
\end{enumerate}
}
\subsection{User Classes and Characteristics}
\emph{
\begin{itemize}
    \item Admin: Responsible for overseeing operations, managing user accounts, monitoring system performance, and resolving issues. Requires a moderate to high level of technical expertise and familiarity with system functionalities.
    \item Delivery Personnel: Primarily uses the product for route optimization, proof of delivery, and real-time updates. Requires basic technical expertise and familiarity with smartphone applications.
    \item Technical Expertise: Includes developers and system administrators responsible for maintaining the application, integrating APIs, and ensuring system reliability. Requires advanced technical knowledge and troubleshooting skills.
    \item Company: Refers to stakeholders or management personnel who utilize delivery insights, performance reports, and other analytics for decision-making. Requires a high-level understanding of operational metrics but minimal technical expertise.
\end{itemize}
}
\subsection{Operating Environment}
\emph{
\begin{itemize}
    \item Hardware Platform:
    \begin{itemize}
        \item Mobile devices with at least 2 GB of RAM and 16 GB of internal storage.
        \item GPS enabled smartphone.
    \end{itemize}
    \item Operating System:
    \begin{itemize}
        \item Android (Version 8.0 and above).
        \item iOS (Version 13.0 and above).
    \end{itemize}
\end{itemize}
}
\subsection{Design and Implementation Constraints}
\emph{
\begin{itemize}
    \item Hardware Limitations: Requires smartphones with GPS functionality, a minimum of 2 GB RAM and stable internet connectivity.
    \item Security Considerations: Data exchanged between the app and server will be encrypted using HTTPS for secure communication.
    \item Programming Standards: The application will be developed using modern frameworks like Flutter for cross-platform compatibility.
    \item Database Technology: A cloud-based database, such as Firebase will store route details, delivery data, and user information.
    \item Third-party APIs: Integration with GPS services for route optimization and location tracking.
    \item Design Standards: Adherence to mobile app design conventions for usability and accessibility.
\end{itemize}
}
\subsection{User Documentation}
\emph{
The following user documentation components will be delivered with the FastTrack application to ensure ease of use and proper understanding of its features:
\begin{itemize}
    \item User Manual:A comprehensive guide in PDF format that covers installation, setup, usage, troubleshooting, and frequently asked questions.
    \item Online Help: Context-sensitive help integrated within the application, offering quick access to tooltips and feature explanations.
    \item Tutorials: Video tutorials hosted on the app, demonstrating key functionalities like route planning and navigation.
    \item Quick Start Guide: A concise, easy-to-follow document that provides a fast overview of the application’s primary features and setup instructions.
    \item Release Notes: A detailed document that outlines new features, updates, and bug fixes for each version, available in PDF format.
\end{itemize}
}
\subsection{Assumptions and Dependencies}
\emph{
Assumptions include:
\begin{itemize}
    \item Users have access to GPS-enabled smartphones with stable internet connectivity.
    \item The application will primarily operate within areas where GPS accuracy and internet speed are reliable.
    \item Delivery routes provided by third-party services are accurate and up-to-date.
\end{itemize}
Dependencies include:
\begin{itemize}
    \item Stable and continuous GPS and internet services for route optimization and real-time updates.
    \item Functional smartphones meeting the specified hardware requirements.
    \item Continued availability and performance of third-party APIs for GPS and location services.
    \item Reliable cloud-based database for storing user data, route details, and delivery updates.
\end{itemize}
} 
\section{System Features}
\subsection{Route Optimization}
\subsubsection{Description and Priority}
\emph{
\begin{itemize}
    \item Description: Curates the best route from an extensive list of destinations (up to 200 per route) and tailored vehicle types.
    \item Priority: High
    \item Benefit: 9
    \item Penalty: 1
    \item Cost: 8
    \item Risk: 3
\end{itemize}
}
\subsubsection{Stimulus/Response Sequences}
\emph{
\begin{itemize}
    \item Stimulus: User inputs a list of destinations.
    \item Response: The system generates the most optimized route based on distance and time, considering current traffic conditions.
\end{itemize}
}
\subsubsection{Functional Requirements}
\emph{
\begin{itemize}
    \item The system should be capable of handling up to 300 stops per route without additional cost.
    \item The route should be optimized using real-time traffic data, considering factors such as time, distance, and mode of delivery.
    \item The system should offer quick optimization for routes with up to 100 stops in 5 seconds.
    \item The user should be able to save and duplicate routes, adding new destinations from previous routes easily.
\end{itemize}
}	
\subsection{Dynamic Route Adjustment}
\subsubsection{Description and Priority}
\emph{
\begin{itemize}
    \item Description: Allows dynamic adjustments to routes in real-time based on changing conditions (e.g., traffic, weather, etc.).
    \item Priority: High
    \item Benefit: 9
    \item Penalty: 1
    \item Cost: 7
    \item Risk: 3
\end{itemize}
}
\subsubsection{Stimulus/Response Sequences}
\emph{
\begin{itemize}
    \item Stimulus: Real-time data indicates a traffic jam or road closure.
    \item Response: The system automatically adjusts the route to avoid delays, presenting the updated route to the driver.
\end{itemize}
}
\subsubsection{Functional Requirements}
\emph{
\begin{itemize}
    \item The system must be capable of adjusting routes in real-time based on live traffic, road closures, and weather conditions.
    \item The system should ensure minimum disruption when adding or removing stops in dynamic routes.
\end{itemize}
}
\subsection{Proof of Delivery and Documentation}
\subsubsection{Description and Priority}
\emph{
\begin{itemize}
    \item Description: Captures proof of delivery via photos, notes, and signatures.
    \item Priority: Medium
    \item Benefit: 7
    \item Penalty: 2
    \item Cost: 5
    \item Risk: 3
\end{itemize}
}
\subsubsection{Stimulus/Response Sequences}
\emph{
\begin{itemize}
    \item Stimulus: User completes a delivery.
    \item Response: The system captures proof (photos, notes, or signatures) for documentation purposes.
\end{itemize}
}
\subsubsection{Functional Requirements}
\emph{
\begin{itemize}
    \item The system should allow users to capture and save photos, notes, and signatures for each stop as proof of delivery.
    \item It must provide the ability to export this proof for further documentation or dispute resolution.
    \item The system should automatically associate proof of delivery with the corresponding stop for easy access.
\end{itemize}
}
\subsection{Dynamic Data-Driven Delivery Insights}
\subsubsection{Description and Priority}
\emph{
\begin{itemize}
    \item Description: Provides detailed statistics on delivery performance, including distance, time, and fuel costs saved.
    \item Priority: Medium
    \item Benefit: 7
    \item Penalty: 4
    \item Cost: 7
    \item Risk: 2
\end{itemize}
}
\subsubsection{Stimulus/Response Sequences}
\emph{
\begin{itemize}
    \item Stimulus: User requests a report or insight into delivery performance.
    \item Response: The system generates detailed reports on delivery efficiency, including statistics on time, distance, and fuel costs.
\end{itemize}
}
\subsubsection{Functional Requirement}
\emph{
\begin{itemize}
    \item The system should provide detailed performance reports on each delivery, including metrics like distance, time, fuel savings, and ETA accuracy.
    \item It must allow users to access and analyze historical route data for performance improvement
\end{itemize}
}
\subsection{Route Export and Data Integration}
\subsubsection{Description and Priority}
\emph{
\begin{itemize}
    \item Description: Facilitates exporting routes in CSV or Excel formats, ensuring compatibility with other systems.
    \item Priority: Medium
    \item Benefit: 7
    \item Penalty: 3
    \item Cost: 6
    \item Risk: 2
\end{itemize}
}
\subsubsection{Stimulus/Response Sequences}
\emph{
\begin{itemize}
    \item Stimulus: User requests an export of the current route data.
    \item Response: The system generates the export in CSV or Excel format for integration with other systems.
\end{itemize}
}
\subsubsection{Functional Requirements}
\emph{
\begin{itemize}
    \item The system should support exporting route data in CSV and Excel formats.
    \item It must ensure compatibility with other logistics and operations management systems.
    \item The export should include all relevant stop data such as addresses, times, phone numbers, and other custom fields.
\end{itemize}
}
\section{External Interface Requirements}
\subsection{User Interfaces}
\emph{
The FastTrack application provides multiple user interfaces to facilitate interactions with the system. The application will feature for mobile interface. Below are the logical characteristics for the interfaces between the software and the users.
\begin{itemize}
    \item Login Screen: The login screen will require users to enter credentials (username/email and password) for authentication. A Forgot Password link will also be provided.
    \item Home Screen: The home screen will display a dashboard with an overview of deliveries, available routes, and important notifications. A navigation bar will allow access to different sections such as Route Optimization, Settings, and User Profile.
    \item Route Optimization Screen: This screen will allow users to input delivery destinations and view the optimized route. It will feature a map interface with a Start Navigation button.
    \item Delivery Details Screen: Once a delivery is selected, users will see detailed information about the delivery, such as destination, estimated time of arrival (ETA), and route breakdown.
\end{itemize}
}
\subsection{Hardware Interfaces}
\emph{
Device Compatibility: The application will support GPS-enabled smartphones and tablets meeting the following minimum requirements:
\begin{itemize}
    \item RAM: 2 GB or more to ensure smooth operation of the application.
    \item Storage: 16 GB free space to accommodate the app and required data, including maps and route optimization data.
    \item Operating Systems: The application will support Android 8.0 (Oreo) or higher, and iOS 12.0 or higher to ensure compatibility with the latest device features and updates.
\end{itemize}
The mobile devices will connect to the internet via Wi-Fi or cellular networks (3G, 4G, or 5G) to communicate with the backend server for route optimization, delivery updates, and real-time notifications.
}
\subsection{Software Interfaces}
\emph{
\begin{itemize}
    \item Operating Systems: Android (version 8.0 or higher) and iOS (version 12.0 or higher).
    \item Cloud-based database: MySQL and Firebase for storing user data and product details.
    \item Third-party APIs: Google Maps API (Android) and Core Location API (iOS) for real-time location tracking and route optimization.
    \item Data exchanged between the app and server  will use JSON format over HTTPS for secure transmission.
\end{itemize}
}
\subsection{Communications Interfaces}
\emph{
\begin{itemize}
    \item Communication Protocols: HTTP/HTTPS for secure client-server communication.
    \item Standards and Security: End-to-end encryption for sensitive data.
    \item Error Handling: Notifications for communication failures.
    \item Synchronization and Transfer Rates: Real-time synchronization for auction updates, product availability, and location tracking and Optimized for low-latency data transfer.
\end{itemize}
}
\section{Other Nonfunctional Requirements}
\subsection{Performance Requirements}
\emph{
\begin{itemize}
    \item Response Time: The system must calculate the optimal route within 2 seconds for graphs with up to 500 nodes and within 5 seconds for larger graphs (up to 2,000 nodes).
    \item Scalability: The system should handle up to 1,000 simultaneous delivery route calculations without significant performance degradation.
    \item Scalability: The system should handle up to 1,000 simultaneous delivery route calculations without significant performance degradation.
    \item Data Update: Real-time traffic updates, if integrated, must be processed within 10 seconds to adjust route calculations dynamically.
    \item Algorithm Efficiency: The implemented algorithm must operate with a time complexity in the range of \( O(n^2) \) to \( O(n^3) \) for graphs where  n is the number of locations (destinations).
\end{itemize}}
\subsection{Safety Requirements}
\emph{
\begin{itemize}
    \item Data Integrity: Ensure no loss or corruption of delivery location data during input or output processing.
    \item Error Handling: Provide clear error messages for invalid inputs.
    \item Backup and Recovery: Include backup mechanisms to save progress during route calculations, ensuring the system can recover in case of crashes.
    \item Regulatory Compliance: Adhere to any data protection regulations if user or customer data is involved in the system.
\end{itemize}
}
\subsection{Security Requirements}
\emph{
\begin{itemize}
    \item Authentication: Require secure login credentials for users accessing the system, especially for systems deployed in enterprise settings.
    \item Privacy: Ensure sensitive data, such as customer addresses or delivery points, is encrypted during transmission and storage.
    \item Access Control: Restrict administrative functionalities to authorized personnel only.
    \item Fraud Prevention: Detect and prevent malicious route manipulations, such as tampering with delivery point data.
\end{itemize}
}
\subsection{Software Quality Attributes}
\emph{
\begin{itemize}
    \item Usability: Provide a simple interface for inputting delivery points and configuring preferences.
    \item Modularity: Implement modular code to simplify debugging and future updates, especially for algorithm enhancements.
    \item Localization: Allow the use of local measurement units and integrate localized traffic data sources if real-time updates are used.
    \item Availability: Implement failover mechanisms to maintain functionality in case of minor hardware or software failures.
    \item Adaptability: Provide support for future updates, including the integration of new mapping services, delivery algorithms, or hardware components, with minimal disruption to existing functionality.
\end{itemize}
}
\section{Other Requirements}
\subsection{Database Requirements}
\emph{
\begin{itemize}
    \item The system must utilize a relational database management system (RDBMS) to store user data, delivery routes, location information, and performance metrics.
    \item The database schema must be scalable to handle an increasing number of deliveries, users, and data points as the system grows.
\end{itemize}
}
\subsection{Internationalization Requirements}
\emph{
\begin{itemize}
    \item Currency and distance units should be configurable according to the user's region.
    \item Date and time formats should be adjustable based on the user's locale to accommodate international users.
\end{itemize}
}
\subsection{Legal Requirements}
\emph{
\begin{itemize}
    \item The application should include clear terms and conditions and privacy policies to inform users about data usage and security measures.
    \item The application must comply with data protection regulations such as the General Data Protection Regulation (GDPR) for users within the EU and Data Protection Laws in India (IT Act, 2000) for users within India.
\end{itemize}
}
\subsection{Backup and Recovery}
\emph{
\begin{itemize}
    \item The system must include automated daily backups of all critical data, ensuring recovery in the event of data loss or corruption.
    \item A disaster recovery plan should be in place to restore full functionality within 24 hours of a major system failure.
\end{itemize}
}
\section{Appendix A: Glossary}
\emph{
\begin{itemize}
    \item API (Application Programming Interface): A set of protocols and tools that allow different software applications to communicate with each other.
    \item GPS (Global Positioning System): A satellite-based navigation system that provides location and time information anywhere on Earth.
    \item RDBMS (Relational Database Management System): A database management system that stores data in a structured format using rows and columns.
    \item Route Optimization: The process of determining the most efficient path or route for delivery based on various factors like distance, time, and traffic.
    \item UI (User Interface): The means by which a user interacts with the system or application, typically through visual elements like buttons, menus, and forms.
    \item API Key: A unique identifier used to authenticate requests to an API service.
    \item ETA (Estimated Time of Arrival): The predicted time a delivery will arrive at a given destination.
    \item GDPR (General Data Protection Regulation): A regulation in the EU law on data protection and privacy in the European Union and the European Economic Area.
    \item Firebase: A platform developed by Google that provides cloud-based backend services like real-time databases, authentication, and hosting for mobile and web applications.
    \item Cloud-based Database: A type of database that is hosted on the cloud, allowing for remote storage and access over the internet.
\end{itemize}
}
\section{Appendix B: Analysis Models}
\emph{
\begin{itemize}
    \item Data Flow Diagram (DFD): A diagram showing how data flows through the system. The DFD will include external entities (users, APIs), processes (route optimization, delivery tracking), data stores (delivery data, user profiles), and data flows.
    \item Use Case Diagram: A visual representation of the system's functional requirements, detailing how users (delivery drivers, administrators) will interact with the system (e.g., input destinations, request optimized routes).
    \item Entity-Relationship Diagram (ERD): A diagram that shows the relationships between the different data entities (e.g., Users, Routes, Deliveries, Locations) within the system.
    \item Class Diagram: A diagram that models the object-oriented structure of the system, showing the classes (e.g., User, Route, Delivery) and their relationships.
\end{itemize}
}

\section{Appendix C: Issues List}
\emph{
\begin{itemize}
    \item TBD: Integration of third-party APIs: The exact details of integrating the Google Maps API and Core Location API still need to be finalized, including the authentication process and API limits.
    \item Data Synchronization: There is an ongoing discussion about how the real-time delivery updates will synchronize across multiple devices and servers.
    \item Traffic Data Accuracy: Ensuring the accuracy and timeliness of real-time traffic data is a challenge, especially in regions with poor coverage or inconsistent data feeds.
    \item User Interface Design: The exact layout and user flow for the route optimization and delivery tracking screens need further refinement to ensure a seamless user experience.
    \item Security for User Data: The security measures for sensitive user data, especially delivery addresses and contact details, are being reviewed to ensure full compliance with GDPR and other data protection laws.
    \item Device Compatibility Issues: Further testing is required to confirm that the app will function properly across all supported devices, especially for different screen sizes and operating systems.
    \item Route Optimization Algorithm Performance: More tests are needed to ensure the optimization algorithm can efficiently handle large datasets and produce accurate routes in real-time.
\end{itemize}
}
\end{document}
